\PassOptionsToPackage{unicode=true}{hyperref} % options for packages loaded elsewhere
\PassOptionsToPackage{hyphens}{url}
%
\documentclass[
]{article}
\usepackage{lmodern}
\usepackage{amssymb,amsmath}
\usepackage{ifxetex,ifluatex}
\ifnum 0\ifxetex 1\fi\ifluatex 1\fi=0 % if pdftex
  \usepackage[T1]{fontenc}
  \usepackage[utf8]{inputenc}
  \usepackage{textcomp} % provides euro and other symbols
\else % if luatex or xelatex
  \usepackage{unicode-math}
  \defaultfontfeatures{Scale=MatchLowercase}
  \defaultfontfeatures[\rmfamily]{Ligatures=TeX,Scale=1}
\fi
% use upquote if available, for straight quotes in verbatim environments
\IfFileExists{upquote.sty}{\usepackage{upquote}}{}
\IfFileExists{microtype.sty}{% use microtype if available
  \usepackage[]{microtype}
  \UseMicrotypeSet[protrusion]{basicmath} % disable protrusion for tt fonts
}{}
\makeatletter
\@ifundefined{KOMAClassName}{% if non-KOMA class
  \IfFileExists{parskip.sty}{%
    \usepackage{parskip}
  }{% else
    \setlength{\parindent}{0pt}
    \setlength{\parskip}{6pt plus 2pt minus 1pt}}
}{% if KOMA class
  \KOMAoptions{parskip=half}}
\makeatother
\usepackage{xcolor}
\IfFileExists{xurl.sty}{\usepackage{xurl}}{} % add URL line breaks if available
\IfFileExists{bookmark.sty}{\usepackage{bookmark}}{\usepackage{hyperref}}
\hypersetup{
  pdftitle={Course Project 1},
  pdfauthor={Alberto Chong},
  pdfborder={0 0 0},
  breaklinks=true}
\urlstyle{same}  % don't use monospace font for urls
\usepackage[margin=1in]{geometry}
\usepackage{color}
\usepackage{fancyvrb}
\newcommand{\VerbBar}{|}
\newcommand{\VERB}{\Verb[commandchars=\\\{\}]}
\DefineVerbatimEnvironment{Highlighting}{Verbatim}{commandchars=\\\{\}}
% Add ',fontsize=\small' for more characters per line
\usepackage{framed}
\definecolor{shadecolor}{RGB}{248,248,248}
\newenvironment{Shaded}{\begin{snugshade}}{\end{snugshade}}
\newcommand{\AlertTok}[1]{\textcolor[rgb]{0.94,0.16,0.16}{#1}}
\newcommand{\AnnotationTok}[1]{\textcolor[rgb]{0.56,0.35,0.01}{\textbf{\textit{#1}}}}
\newcommand{\AttributeTok}[1]{\textcolor[rgb]{0.77,0.63,0.00}{#1}}
\newcommand{\BaseNTok}[1]{\textcolor[rgb]{0.00,0.00,0.81}{#1}}
\newcommand{\BuiltInTok}[1]{#1}
\newcommand{\CharTok}[1]{\textcolor[rgb]{0.31,0.60,0.02}{#1}}
\newcommand{\CommentTok}[1]{\textcolor[rgb]{0.56,0.35,0.01}{\textit{#1}}}
\newcommand{\CommentVarTok}[1]{\textcolor[rgb]{0.56,0.35,0.01}{\textbf{\textit{#1}}}}
\newcommand{\ConstantTok}[1]{\textcolor[rgb]{0.00,0.00,0.00}{#1}}
\newcommand{\ControlFlowTok}[1]{\textcolor[rgb]{0.13,0.29,0.53}{\textbf{#1}}}
\newcommand{\DataTypeTok}[1]{\textcolor[rgb]{0.13,0.29,0.53}{#1}}
\newcommand{\DecValTok}[1]{\textcolor[rgb]{0.00,0.00,0.81}{#1}}
\newcommand{\DocumentationTok}[1]{\textcolor[rgb]{0.56,0.35,0.01}{\textbf{\textit{#1}}}}
\newcommand{\ErrorTok}[1]{\textcolor[rgb]{0.64,0.00,0.00}{\textbf{#1}}}
\newcommand{\ExtensionTok}[1]{#1}
\newcommand{\FloatTok}[1]{\textcolor[rgb]{0.00,0.00,0.81}{#1}}
\newcommand{\FunctionTok}[1]{\textcolor[rgb]{0.00,0.00,0.00}{#1}}
\newcommand{\ImportTok}[1]{#1}
\newcommand{\InformationTok}[1]{\textcolor[rgb]{0.56,0.35,0.01}{\textbf{\textit{#1}}}}
\newcommand{\KeywordTok}[1]{\textcolor[rgb]{0.13,0.29,0.53}{\textbf{#1}}}
\newcommand{\NormalTok}[1]{#1}
\newcommand{\OperatorTok}[1]{\textcolor[rgb]{0.81,0.36,0.00}{\textbf{#1}}}
\newcommand{\OtherTok}[1]{\textcolor[rgb]{0.56,0.35,0.01}{#1}}
\newcommand{\PreprocessorTok}[1]{\textcolor[rgb]{0.56,0.35,0.01}{\textit{#1}}}
\newcommand{\RegionMarkerTok}[1]{#1}
\newcommand{\SpecialCharTok}[1]{\textcolor[rgb]{0.00,0.00,0.00}{#1}}
\newcommand{\SpecialStringTok}[1]{\textcolor[rgb]{0.31,0.60,0.02}{#1}}
\newcommand{\StringTok}[1]{\textcolor[rgb]{0.31,0.60,0.02}{#1}}
\newcommand{\VariableTok}[1]{\textcolor[rgb]{0.00,0.00,0.00}{#1}}
\newcommand{\VerbatimStringTok}[1]{\textcolor[rgb]{0.31,0.60,0.02}{#1}}
\newcommand{\WarningTok}[1]{\textcolor[rgb]{0.56,0.35,0.01}{\textbf{\textit{#1}}}}
\usepackage{graphicx,grffile}
\makeatletter
\def\maxwidth{\ifdim\Gin@nat@width>\linewidth\linewidth\else\Gin@nat@width\fi}
\def\maxheight{\ifdim\Gin@nat@height>\textheight\textheight\else\Gin@nat@height\fi}
\makeatother
% Scale images if necessary, so that they will not overflow the page
% margins by default, and it is still possible to overwrite the defaults
% using explicit options in \includegraphics[width, height, ...]{}
\setkeys{Gin}{width=\maxwidth,height=\maxheight,keepaspectratio}
\setlength{\emergencystretch}{3em}  % prevent overfull lines
\providecommand{\tightlist}{%
  \setlength{\itemsep}{0pt}\setlength{\parskip}{0pt}}
\setcounter{secnumdepth}{-2}
% Redefines (sub)paragraphs to behave more like sections
\ifx\paragraph\undefined\else
  \let\oldparagraph\paragraph
  \renewcommand{\paragraph}[1]{\oldparagraph{#1}\mbox{}}
\fi
\ifx\subparagraph\undefined\else
  \let\oldsubparagraph\subparagraph
  \renewcommand{\subparagraph}[1]{\oldsubparagraph{#1}\mbox{}}
\fi

% set default figure placement to htbp
\makeatletter
\def\fps@figure{htbp}
\makeatother


\title{Course Project 1}
\author{Alberto Chong}
\date{12/29/2020}

\begin{document}
\maketitle

\hypertarget{libraries}{%
\subsubsection{libraries}\label{libraries}}

We will attach the following libraries:

\begin{Shaded}
\begin{Highlighting}[]
\KeywordTok{library}\NormalTok{(dplyr)}
\KeywordTok{library}\NormalTok{(ggplot2)}
\KeywordTok{library}\NormalTok{(tidyr)}
\end{Highlighting}
\end{Shaded}

\hypertarget{loading-and-preprocessing-the-data}{%
\subsubsection{1. Loading and preprocessing the
data}\label{loading-and-preprocessing-the-data}}

\hypertarget{loading-data}{%
\paragraph{Loading data}\label{loading-data}}

\begin{Shaded}
\begin{Highlighting}[]
\NormalTok{activity <-}\StringTok{ }\KeywordTok{read.csv}\NormalTok{(}\KeywordTok{unzip}\NormalTok{(}\KeywordTok{paste0}\NormalTok{(}\KeywordTok{getwd}\NormalTok{(),}\StringTok{"/activity.zip"}\NormalTok{)))}
\end{Highlighting}
\end{Shaded}

\hypertarget{transform-data}{%
\paragraph{transform data}\label{transform-data}}

\begin{Shaded}
\begin{Highlighting}[]
\NormalTok{activity <-}\StringTok{ }\NormalTok{activity }\OperatorTok\StringTok{ }\KeywordTok{mutate}\NormalTok{(}\DataTypeTok{date =} \KeywordTok{as.Date}\NormalTok{(date))}
\KeywordTok{str}\NormalTok{(activity)}
\end{Highlighting}
\end{Shaded}

\begin{verbatim}
## 'data.frame':    17568 obs. of  3 variables:
##  $ steps   : int  NA NA NA NA NA NA NA NA NA NA ...
##  $ date    : Date, format: "2012-10-01" "2012-10-01" ...
##  $ interval: int  0 5 10 15 20 25 30 35 40 45 ...
\end{verbatim}

\hypertarget{what-is-mean-total-number-of-steps-taken-per-day}{%
\subsubsection{2. What is mean total number of steps taken per
day?}\label{what-is-mean-total-number-of-steps-taken-per-day}}

\hypertarget{calculate-the-total-number-of-steps-taken-per-day}{%
\paragraph{Calculate the total number of steps taken per
day}\label{calculate-the-total-number-of-steps-taken-per-day}}

\begin{Shaded}
\begin{Highlighting}[]
\NormalTok{Total <-}\StringTok{ }\NormalTok{activity }\OperatorTok\StringTok{ }
\StringTok{    }\KeywordTok{group_by}\NormalTok{(date) }\OperatorTok\StringTok{ }
\StringTok{    }\KeywordTok{summarise}\NormalTok{(}\DataTypeTok{total =} \KeywordTok{sum}\NormalTok{(steps,}\DataTypeTok{na.rm =} \OtherTok{TRUE}\NormalTok{)) }\OperatorTok\StringTok{ }
\StringTok{    }\KeywordTok{ungroup}\NormalTok{()}
\KeywordTok{head}\NormalTok{(Total)}
\end{Highlighting}
\end{Shaded}

\begin{verbatim}
## # A tibble: 6 x 2
##   date       total
##   <date>     <int>
## 1 2012-10-01     0
## 2 2012-10-02   126
## 3 2012-10-03 11352
## 4 2012-10-04 12116
## 5 2012-10-05 13294
## 6 2012-10-06 15420
\end{verbatim}

\hypertarget{make-a-histogram-of-the-total-number-of-steps-taken-each-day}{%
\paragraph{Make a histogram of the total number of steps taken each
day}\label{make-a-histogram-of-the-total-number-of-steps-taken-each-day}}

\begin{Shaded}
\begin{Highlighting}[]
\KeywordTok{hist}\NormalTok{(Total}\OperatorTok{$}\NormalTok{total,}
     \DataTypeTok{main =} \StringTok{"Histogram"}\NormalTok{, }
     \DataTypeTok{xlab =} \StringTok{"steps per day"}\NormalTok{)}
\end{Highlighting}
\end{Shaded}

\includegraphics{PA1_template_files/figure-latex/unnamed-chunk-2-1.pdf}

\hypertarget{calculate-and-report-the-mean-and-median-of-the-total-number-of-steps-taken-per-day}{%
\subsubsection{3. Calculate and report the mean and median of the total
number of steps taken per
day}\label{calculate-and-report-the-mean-and-median-of-the-total-number-of-steps-taken-per-day}}

Mean of total number of steps per day

\begin{Shaded}
\begin{Highlighting}[]
\KeywordTok{mean}\NormalTok{(Total}\OperatorTok{$}\NormalTok{total)}
\end{Highlighting}
\end{Shaded}

\begin{verbatim}
## [1] 9354.23
\end{verbatim}

Median of total number of steps

\begin{Shaded}
\begin{Highlighting}[]
\KeywordTok{median}\NormalTok{(Total}\OperatorTok{$}\NormalTok{total)}
\end{Highlighting}
\end{Shaded}

\begin{verbatim}
## [1] 10395
\end{verbatim}

\hypertarget{what-is-the-average-daily-activity-pattern}{%
\subsubsection{4. What is the average daily activity
pattern?}\label{what-is-the-average-daily-activity-pattern}}

\hypertarget{time-series-plot}{%
\paragraph{Time series plot}\label{time-series-plot}}

\begin{Shaded}
\begin{Highlighting}[]
\NormalTok{activity }\OperatorTok\StringTok{ }\KeywordTok{group_by}\NormalTok{(interval) }\OperatorTok\StringTok{ }
\StringTok{  }\KeywordTok{summarise}\NormalTok{(}\DataTypeTok{avg_steps =} \KeywordTok{mean}\NormalTok{(steps,}\DataTypeTok{na.rm =}\NormalTok{ T)) }\OperatorTok\StringTok{ }
\StringTok{  }\KeywordTok{ggplot}\NormalTok{(}\KeywordTok{aes}\NormalTok{(}\DataTypeTok{x =}\NormalTok{ interval, }\DataTypeTok{y =}\NormalTok{ avg_steps)) }\OperatorTok{+}\StringTok{ }
\StringTok{  }\KeywordTok{geom_line}\NormalTok{()}
\end{Highlighting}
\end{Shaded}

\includegraphics{PA1_template_files/figure-latex/timeseries-1.pdf}

\hypertarget{minute-interval-that-contains-the-maximum-number-of-steps-on-average}{%
\subsubsection{5. 5-minute interval that contains the maximum number of
steps on
average}\label{minute-interval-that-contains-the-maximum-number-of-steps-on-average}}

\begin{Shaded}
\begin{Highlighting}[]
\NormalTok{avg_interval <-}\StringTok{ }\NormalTok{activity }\OperatorTok\StringTok{ }\KeywordTok{group_by}\NormalTok{(interval) }\OperatorTok\StringTok{ }
\StringTok{    }\KeywordTok{summarise}\NormalTok{(}\DataTypeTok{avg_steps =} \KeywordTok{mean}\NormalTok{(steps,}\DataTypeTok{na.rm =}\NormalTok{ T))}
\NormalTok{avg_interval[}\KeywordTok{which}\NormalTok{(avg_interval}\OperatorTok{$}\NormalTok{avg_steps }\OperatorTok{==}\StringTok{ }\KeywordTok{max}\NormalTok{(avg_interval}\OperatorTok{$}\NormalTok{avg_steps,}\DataTypeTok{na.rm =}\NormalTok{ T)),]}
\end{Highlighting}
\end{Shaded}

\begin{verbatim}
## # A tibble: 1 x 2
##   interval avg_steps
##      <int>     <dbl>
## 1      835      206.
\end{verbatim}

\hypertarget{imputing-missing-values}{%
\subsubsection{6. Imputing missing
values}\label{imputing-missing-values}}

\hypertarget{calculate-and-report-the-total-number-of-missing-values-in-the-dataset}{%
\paragraph{Calculate and report the total number of missing values in
the
dataset}\label{calculate-and-report-the-total-number-of-missing-values-in-the-dataset}}

\begin{Shaded}
\begin{Highlighting}[]
\KeywordTok{sum}\NormalTok{(}\KeywordTok{is.na}\NormalTok{(activity}\OperatorTok{$}\NormalTok{steps))}
\end{Highlighting}
\end{Shaded}

\begin{verbatim}
## [1] 2304
\end{verbatim}

\hypertarget{strategy-for-filling-in-all-of-the-missing-values-in-the-dataset}{%
\paragraph{Strategy for filling in all of the missing values in the
dataset}\label{strategy-for-filling-in-all-of-the-missing-values-in-the-dataset}}

\begin{Shaded}
\begin{Highlighting}[]
\NormalTok{avg_day <-}\StringTok{ }\NormalTok{activity }\OperatorTok\StringTok{ }
\StringTok{    }\KeywordTok{group_by}\NormalTok{(date) }\OperatorTok\StringTok{ }
\StringTok{    }\KeywordTok{summarise}\NormalTok{(}\DataTypeTok{avg =} \KeywordTok{round}\NormalTok{(}\KeywordTok{mean}\NormalTok{(steps, }\DataTypeTok{na.rm =} \OtherTok{TRUE}\NormalTok{),}\DecValTok{0}\NormalTok{)) }\OperatorTok\StringTok{ }
\StringTok{    }\KeywordTok{ungroup}\NormalTok{() }\OperatorTok\StringTok{ }
\StringTok{    }\KeywordTok{replace_na}\NormalTok{(}\KeywordTok{list}\NormalTok{(}\DataTypeTok{avg =} \DecValTok{0}\NormalTok{))}
\KeywordTok{head}\NormalTok{(avg_day)}
\end{Highlighting}
\end{Shaded}

\begin{verbatim}
## # A tibble: 6 x 2
##   date         avg
##   <date>     <dbl>
## 1 2012-10-01     0
## 2 2012-10-02     0
## 3 2012-10-03    39
## 4 2012-10-04    42
## 5 2012-10-05    46
## 6 2012-10-06    54
\end{verbatim}

\hypertarget{create-a-new-dataset-that-is-equal-to-the-original-dataset-but-with-the-missing-data-filled-in}{%
\paragraph{Create a new dataset that is equal to the original dataset
but with the missing data filled
in}\label{create-a-new-dataset-that-is-equal-to-the-original-dataset-but-with-the-missing-data-filled-in}}

\begin{Shaded}
\begin{Highlighting}[]
\NormalTok{imputed_activity <-}\StringTok{ }\NormalTok{activity }\OperatorTok\StringTok{ }
\StringTok{    }\KeywordTok{inner_join}\NormalTok{(avg_day, }\DataTypeTok{by =} \StringTok{"date"}\NormalTok{) }\OperatorTok\StringTok{ }
\StringTok{    }\KeywordTok{mutate}\NormalTok{(}\DataTypeTok{steps =} \KeywordTok{ifelse}\NormalTok{(}\KeywordTok{is.na}\NormalTok{(steps),avg,steps)) }\OperatorTok\StringTok{ }
\StringTok{    }\KeywordTok{select}\NormalTok{(}\OperatorTok{-}\NormalTok{avg)}

\KeywordTok{head}\NormalTok{(imputed_activity)}
\end{Highlighting}
\end{Shaded}

\begin{verbatim}
##   steps       date interval
## 1     0 2012-10-01        0
## 2     0 2012-10-01        5
## 3     0 2012-10-01       10
## 4     0 2012-10-01       15
## 5     0 2012-10-01       20
## 6     0 2012-10-01       25
\end{verbatim}

\hypertarget{make-a-histogram-of-the-total-number-of-steps-taken-each-day-and-calculate-and-report-the-mean-and-median-total-number-of-steps-taken-per-day}{%
\subsubsection{7. Make a histogram of the total number of steps taken
each day and Calculate and report the mean and median total number of
steps taken per
day}\label{make-a-histogram-of-the-total-number-of-steps-taken-each-day-and-calculate-and-report-the-mean-and-median-total-number-of-steps-taken-per-day}}

Histogram

\begin{Shaded}
\begin{Highlighting}[]
\NormalTok{imputed_Total <-}\StringTok{ }\NormalTok{imputed_activity }\OperatorTok\StringTok{ }
\StringTok{    }\KeywordTok{group_by}\NormalTok{(date) }\OperatorTok\StringTok{ }
\StringTok{    }\KeywordTok{summarise}\NormalTok{(}\DataTypeTok{total =} \KeywordTok{sum}\NormalTok{(steps,}\DataTypeTok{na.rm =} \OtherTok{TRUE}\NormalTok{)) }\OperatorTok\StringTok{ }
\StringTok{    }\KeywordTok{ungroup}\NormalTok{()}
\KeywordTok{hist}\NormalTok{(imputed_Total}\OperatorTok{$}\NormalTok{total,}
     \DataTypeTok{main =} \StringTok{"Histogram (imputed data)"}\NormalTok{, }
     \DataTypeTok{xlab =} \StringTok{"steps per day"}\NormalTok{)}
\end{Highlighting}
\end{Shaded}

\includegraphics{PA1_template_files/figure-latex/unnamed-chunk-9-1.pdf}

Mean:

\begin{Shaded}
\begin{Highlighting}[]
\KeywordTok{mean}\NormalTok{(imputed_Total}\OperatorTok{$}\NormalTok{total)}
\end{Highlighting}
\end{Shaded}

\begin{verbatim}
## [1] 9354.23
\end{verbatim}

Median:

\begin{Shaded}
\begin{Highlighting}[]
\KeywordTok{median}\NormalTok{(imputed_Total}\OperatorTok{$}\NormalTok{total)}
\end{Highlighting}
\end{Shaded}

\begin{verbatim}
## [1] 10395
\end{verbatim}

The values of mean an median didn't change from the estimates calculated
from the first part of the assigment.

\hypertarget{are-there-differences-in-activity-patterns-between-weekdays-and-weekends}{%
\subsubsection{8. Are there differences in activity patterns between
weekdays and
weekends?}\label{are-there-differences-in-activity-patterns-between-weekdays-and-weekends}}

\hypertarget{create-a-new-factor-variable-in-the-dataset-with-two-levels---weekday-and-weekend}{%
\paragraph{Create a new factor variable in the dataset with two levels -
``weekday'' and
``weekend''}\label{create-a-new-factor-variable-in-the-dataset-with-two-levels---weekday-and-weekend}}

\begin{Shaded}
\begin{Highlighting}[]
\NormalTok{imputed_activity}\OperatorTok{$}\NormalTok{weekday <-}\StringTok{ }\KeywordTok{weekdays}\NormalTok{(imputed_activity}\OperatorTok{$}\NormalTok{date)}
\NormalTok{imputed_activity <-}\StringTok{ }\NormalTok{imputed_activity }\OperatorTok\StringTok{ }
\StringTok{    }\KeywordTok{mutate}\NormalTok{(}\DataTypeTok{class_weekday =} \KeywordTok{ifelse}\NormalTok{(weekday }\OperatorTok\StringTok{ }\KeywordTok{c}\NormalTok{(}\StringTok{"Saturday"}\NormalTok{,}\StringTok{"Sunday"}\NormalTok{),}\StringTok{"weekend"}\NormalTok{,}\StringTok{"weekday"}\NormalTok{))}
\KeywordTok{head}\NormalTok{(imputed_activity)}
\end{Highlighting}
\end{Shaded}

\begin{verbatim}
##   steps       date interval weekday class_weekday
## 1     0 2012-10-01        0  Monday       weekday
## 2     0 2012-10-01        5  Monday       weekday
## 3     0 2012-10-01       10  Monday       weekday
## 4     0 2012-10-01       15  Monday       weekday
## 5     0 2012-10-01       20  Monday       weekday
## 6     0 2012-10-01       25  Monday       weekday
\end{verbatim}

\hypertarget{make-a-panel-plot-containing-a-time-series-plot}{%
\paragraph{Make a panel plot containing a time series
plot}\label{make-a-panel-plot-containing-a-time-series-plot}}

\begin{Shaded}
\begin{Highlighting}[]
\NormalTok{imputed_activity }\OperatorTok\StringTok{ }
\StringTok{  }\KeywordTok{group_by}\NormalTok{(interval,class_weekday) }\OperatorTok\StringTok{ }
\StringTok{  }\KeywordTok{summarise}\NormalTok{(}\DataTypeTok{avg_steps =} \KeywordTok{mean}\NormalTok{(steps,}\DataTypeTok{na.rm =}\NormalTok{ T)) }\OperatorTok\StringTok{ }
\StringTok{  }\KeywordTok{ggplot}\NormalTok{(}\KeywordTok{aes}\NormalTok{(}\DataTypeTok{x =}\NormalTok{ interval, }\DataTypeTok{y =}\NormalTok{ avg_steps)) }\OperatorTok{+}\StringTok{ }
\StringTok{  }\KeywordTok{geom_line}\NormalTok{() }\OperatorTok{+}\StringTok{ }
\StringTok{  }\KeywordTok{facet_wrap}\NormalTok{(}\OperatorTok{~}\NormalTok{class_weekday, }\DataTypeTok{dir =} \StringTok{"v"}\NormalTok{)}
\end{Highlighting}
\end{Shaded}

\includegraphics{PA1_template_files/figure-latex/unnamed-chunk-13-1.pdf}

\end{document}
